\section{Discussion}

Thrombolysis use in England and Wales is highly variable between stroke teams, and falls well short of NHS England's target of 20\% emergency stroke admissions, and is much lower overall than the European average (15\% of all stroke \cite{aguiar_de_sousa_delivery_2023}). Previously we have identified that hospital processes and decision-making cause more variation than differences in patient populations. A significant amount of variation comes from clinician's enthusiasm to use thrombolysis \cite{allen_use_2022, allen_using_2022}. In that earlier work, we developed methods to compare decision-making between stroke teams. We subsequently added an explainability model to the predictions \cite{pearn_what_2023}, and we were able to identify factors that affected choice of thrombolysis and showed that almost all stroke teams would normally give thrombolysis to \textit{ideal} candidates for thrombolysis (e.g. a patient with moderate-severe stroke, no prior disability, thrombolysis would given in 2 hours, age under 80) but stroke teams varied in their use of thrombolysis in patients with non-ideal characteristics. In associated qualitative work, we found that while clinicians would be influenced by understanding the practice of other clinicians, they also wanted the machine learning modelling expanded to predict outcomes to ensure that higher thrombolysing units were doing more good than harm \cite{allen_using_2022}. In the work we describe here we have addressed outcomes. The work described here is companion work to in-depth studies on how the observed effect of thrombolysis compares with clinical trials \cite{pearn_thrombolysis_2024} and a study on whether those receiving thrombolysis are the the ones who will most likely benefit from it \cite{pearn_are_2024}. Using simplified models compared to our first work, we were also able to construct \textit{prototype patients} to exemplify variation in thrombolysis use and outcomes. We also used the disability-level outcomes of the machine learning model as the input to a health economics model to evaluate the cost-effectiveness of thrombolysis from observational data on use of thrombolysis.

\textit{Prototype patients} showed how non-ideal characteristics of a candidate for thrombolysis affected stroke teams' willingness to use thrombolysis. The largest variation in use came with mild stroke patients and those patients with prior disability. When considering overall use of thrombolysis, use in mild stroke patients will be very influential as mild stroke (NIHSS 0-4) make up 53\% of the general emergency stroke population, though they made up less than 10\% of the clinical trial population \cite{emberson_effect_2014}. Our analysis showed that patients with mild stroke could still benefit from thrombolysis, but the benefit is smaller than in more severe stroke (as there is less room for improvement) and is dependent on other patient features. An analysis of admitted mild stroke patients (rather than prototype patients) showed that those mild stroke patients actually receiving thrombolysis are predicted to have improved outcomes when measured as the proportion of mRS 0-1 or mRS 0-2, or the average disability outcome, but with a small increase (no more than 1\% absolute risk increase) in risk of the worst outcomes (mRS 5-6). Other studies \cite{romano_predictors_2021, coutts_tenecteplase_2024} reported no net benefit, or net harm, of thrombolysis in patients with mild stroke (NIHSS 0-5). Our models predict benefit is present in some, but not all, patients with mild stroke, and overall those receiving thrombolysis (or would receive thrombolysis at the \textit{benchmark stroke teams} with higher thrombolysis use), are receiving a net improvement in outcomes (measured as shift in mRS).  Further work to investigate and inform guidelines around use of thrombolysis in mild stroke could be beneficial.

Our models also indicate that teams also varied in their attitude to use of thrombolysis in patients with prior disability. Though the existence of prior disability limits the improvement possible with thrombolysis, our machine learning identified that there was still significant benefit to be had from thrombolysis. Patients with significant prior disability were excluded from the landmark trials of thrombolysis, but our outcome-based analysis of a large, prospective real-world observational registry would suggest that these patients stand to benefit to the same extent as those without prior disability - reassuring clinicians that this extension to the strict eligibility criteria of the original trials is still resulting in substantial net benefit and cost-effectiveness. 

A third area where stroke teams differ is willingness to use thrombolysis in patients with an estimated rather than precise onset time. Our modelling suggests that outcomes when the onset time is estimated, rather than known precisely, is the same or slightly better (for any assumed onset-to-thrombolysis time) than with a precisely known onset time. The slight improvement would be explained by a cautious estimation of stroke onset time, meaning that onset-to-treatment time is more likely to be over-estimated than under-estimated. It is notable that this factor impedes use of thrombolysis in some stroke teams, but is not detrimental to outcomes from thrombolysis. This finding, derived from a large national stroke registry, should increase the confidence with which teams could proceed with thrombolysis in patients where onset-to-treatment time is based on a `best estimate' rather than known precisely or witnessed.

We have previously developed the concept of \textit{benchmark stroke teams} and \textit{benchmark thrombolysis decisions}. \textit{Benchmark stroke teams} are the teams most willing to use thrombolysis if all teams saw the same group of patients (representative of the patient population). \textit{Benchmark thrombolysis decisions} are decisions that would be likely made for any patient if the majority vote of the \textit{benchmark stroke teams} was followed. A question remained, arising from our companion qualitative work \cite{jarvie_stroke_2024}, whether benchmark stroke teams were gaining more benefit than other teams, or were over-using thrombolysis leading to more harm than good. In our analysis here we show that benchmark stroke teams are expected to be providing more net benefit - they are maintaining the benefit of thrombolysis in the treatment group but are expanding that benefit to more patients. Looking at decision-making in these higher thrombolysing teams may give lower-thrombolysing teams confidence to broaden their own use of thrombolysis.

Our health economics predictions, based on machine learning predicting disability at discharge with and without thrombolysis estimated that the QALY gained for each use of thrombolysis was 0.24, very similar to calculations made by earlier work from SSNAP, which estimated 0.26 QALYs gained for each patient treated with thrombolysis \cite{sentinel_stroke_national_audit_programme_cost_2016}. From our observational analysis thrombolysis offers good cost per QALY in direct treatment costs and is predicted to save NHS in healthcare costs even when applied to a much wider group of patients than originally described in the efficacy trials.

A large focus of our work has been on understanding thrombolysis decisions and outcomes. The full pathway model, however, highlights other areas of interest that we have identified before. Stroke teams vary in the proportion of patients where stroke onset time is determined. While some of this variation is unavoidable it is very possible some variation comes from in-hospital processes (which may be affected by enthusiasm to use thrombolysis). Speed, and consistency of speed, of the thrombolysis pathway also needs to remain a focus. While improving speed only changes use of thrombolysis in those patients that move from out-of-time to in-time windows of thrombolysis use, faster pathway speeds benefit all who are receiving thrombolysis. The loss of effectiveness of thrombolysis over time was well established in clinical trials \cite{emberson_effect_2014}. We have found the same overall decay in effectiveness of thrombolysis over time, but also found the decay in effectiveness in more severe stroke (those with most benefit from thrombolysis) is steeper than the decay in milder stroke \cite{pearn_thrombolysis_2024}, so time and speed of thrombolysis will always be important to all patients, with small improvements benefiting all those treated \cite{meretoja_stroke_2014}. This benefit is also seen from improvements in ambulance response and on-scene times - our modelling predicts a larger effect on outcomes than on thrombolysis use as, again, all patients including those who would have arrived in time for thrombolysis anyway, benefit from the faster onset-to-treatment times. 

Pathway analysis may be conducted at stroke team level. This will enable stroke teams to identify which part of the thrombolysis pathway is most likely to influence use and benefit from thrombolysis. For some it will be decision-making around thrombolysis, for others it will be process speeds and consistency, and for others it will be determination of stroke onset times. Team pathway analysis will also provide a bespoke and achievable thrombolysis use target for each stroke team, taking into account their own local patient population.

\subsection{Study limitations and further work}

Our model is limited to data available in SSNAP, although we have previously demonstrated that nearly all the observed variation in thrombolysis rates between sites is accounted for by factors measured in SSNAP. Although the accuracy overall is very good (ROC-AUC of at least 0.8 for the machine learning models predicting thrombolysis use and outcome after stroke), our models are not intended for individual clinical decision-making. Machine learning is at its most effective when considering overall patterns present in the data (which then provides knowledge that can be passed to clinicians, without the models then always being needed to be run). In the absence of individual patient-level predictions, we suggest future work should focus on providing more sophisticated guidance (though without requiring specialist models) on selection of patients for thrombolysis, especially helping to inform clinicians on which patients with mild stroke who are likely to receive benefit from thrombolysis.

This consideration will particularly apply in the subset of patients with a posterior circulation stroke, for whom the NIHSS underscores severity and is less strongly related to outcomes than in anterior circulation stroke. More generally, we have used NIHSS as a surrogate for the disabling potential of a stroke, when the distinction between disabling/non-disabling stroke should be preferably made from the patient's unique standpoint, rather than dichotomising at an arbitrary threshold of NIHSS \cite{braksick_thrombolysis_2024}. There is also significant scope to use the same techniques to study variation in use of thrombectomy, and how that variation affects patient outcomes.